\documentclass{KNED}
\usepackage{amsmath}

\renewcommand{\KNEDLHdr}{Emad Pourhassani}
\renewcommand{\KNEDRHdr}{Theory Of Computation}

\begin{document}

\begin{exercise}[Context Free Languages:]
Consider the language \( L = \{0^n1^n \mid n \geq 0\} \).

\begin{enumerate}
    \item Prove that \( L \) is context-free.
    \item Construct a context-free grammar for \( L \).
\end{enumerate}

\end{exercise}

\begin{solution}

\begin{enumerate}
    \item To prove that \( L \) is context-free, we can construct a pushdown automaton (PDA) that recognizes it. The PDA's stack will keep track of the number of \(0\)s encountered before encountering \(1\)s. For each \(0\) encountered, we push a symbol onto the stack, and for each \(1\) encountered, we pop a symbol from the stack. We accept if the stack is empty when all symbols are read. Since we can construct a PDA for \( L \), \( L \) is context-free.
    
    \item A context-free grammar \( G \) for \( L \) can be defined as follows:
    \begin{align*}
        S &\rightarrow 0S1 \\
        S &\rightarrow \varepsilon
    \end{align*}
    where \( S \) is the start symbol, and \( \varepsilon \) represents the empty string. These production rules generate strings of the form \(0^n1^n\), where \(n \geq 0\). So, the context-free grammar \( G \) generates the language \( L = \{0^n1^n \mid n \geq 0\} \).
\end{enumerate}
\end{solution}

\end{document}

