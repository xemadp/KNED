\documentclass[english]{KNED}

\usepackage{hyperref}
%\usepackage{xepersian}

\renewcommand{\KNEDLHdr}{Group Theory}
\renewcommand{\KNEDRHdr}{}

\begin{document}

% DEFINITION ENVIRONMENT
\begin{definition}
A \textbf{group} is a set $G$ with an operation $\cdot$ such that:
\begin{enumerate}
    \item $G$ is closed under the operation.
    \item The operation is associative.
    \item There is an identity element.
    \item Every element has an inverse.
\end{enumerate}
\end{definition}

% COROLLARY ENVIRONMENT
\begin{corollary}
The identity element is unique in any group.
\end{corollary}

% REMARK ENVIRONMENT
\begin{remark}
This result follows from the definition of groups.
\end{remark}

% EXAMPLE ENVIRONMENT
\begin{example}
Consider the group $(\mathbb{Z}, +)$, the set of integers under addition. The identity element is $0$, and each element $z \in \mathbb{Z}$ has an inverse $-z$.
\end{example}


% EXERCISE ENVIRONMENT
\begin{exercise}
Prove the following theorem about groups.
\begin{theorem}
No group can be written down as the union of two of its proper subgroups.
\end{theorem}

\end{exercise}

% SOLUTION ENVIRONMENT
\begin{solution}
We provide a proof by contradiction.

% PROOF ENVIRONMENT
\begin{proof}
Assume two proper subgroups of group $G$, call them $H_1$ and $H_2$.\\ 
Assume $H_1 \cup H_2 = G$. We know that $e \in H_1 \cap H_2$, so $e \in H_1 \cup H_2$.\\
Let $h_1 \in G$ such that $h_1 \in H_1 \setminus H_2$.\\
Since $H_1 \neq H_2 \neq G \neq \emptyset$, there exists $h_1$ only in $H_1 \setminus H_2$. Similarly, $h_2 \in G$ exists such that $h_2 \in H_2 \setminus H_1$.\\
If $h_1h_2 \in H_1$, then $h_2 \in H_1$, which contradicts our assumption. Therefore, $H_1 \cup H_2 \neq G$.
\end{proof}

\end{solution}


\end{document}
