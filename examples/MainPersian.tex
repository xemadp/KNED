\documentclass[persian]{KNED}

\usepackage{hyperref}
\usepackage{xepersian}

\settextfont{Amiri}
\renewcommand{\KNEDLHdr}{}
\renewcommand{\KNEDRHdr}{نظریه گروه ها}

\begin{document}

% DEFINITION ENVIRONMENT
\begin{definition}
یک \textbf{گروه} مجموعه‌ای است $G$ با یک عملگر $\cdot$ به‌گونه‌ای که:
\begin{enumerate}
    \item $G$ نسبت به این عملگر بسته است.
    \item این عملگر دارای خاصیت شرکت‌پذیری است.
    \item یک عنصر همانی وجود دارد.
    \item هر عنصر دارای معکوس است.
\end{enumerate}
\end{definition}

% COROLLARY ENVIRONMENT
\begin{corollary}
عنصر همانی در هر گروه یکتا است.
\end{corollary}

% REMARK ENVIRONMENT
\begin{remark}
این نتیجه از تعریف گروه‌ها حاصل می‌شود.
\end{remark}

% EXAMPLE ENVIRONMENT
\begin{example}
گروه $(\mathbb{Z}, +)$ را در نظر بگیرید، مجموعه اعداد صحیح تحت عمل جمع. عنصر همانی $0$ است و هر عنصر $z \in \mathbb{Z}$ دارای معکوس $-z$ است.
\end{example}


% EXERCISE ENVIRONMENT
\begin{exercise}
ثابت کنید قضیه زیر در مورد گروه‌ها.
\begin{theorem}
هیچ گروهی نمی‌تواند به عنوان اجتماع دو زیرگروه خاص خود نوشته شود.
\end{theorem}

\end{exercise}

% SOLUTION ENVIRONMENT
\begin{solution}
ما با استفاده از برهان خلف، اثبات را ارائه می‌دهیم.

% PROOF ENVIRONMENT
\begin{proof}
فرض کنید دو زیرگروه خاص از گروه $G$ داریم، آنها را $H_1$ و $H_2$ بنامیم.\\ 
فرض کنید $H_1 \cup H_2 = G$. می‌دانیم که $e \in H_1 \cap H_2$، بنابراین $e \in H_1 \cup H_2$.\\
بگذارید $h_1 \in G$ به‌گونه‌ای که $h_1 \in H_1 \setminus H_2$.\\
چون $H_1 \neq H_2 \neq G \neq \emptyset$، عنصری مانند $h_1$ تنها در $H_1 \setminus H_2$ وجود دارد. به همین ترتیب، عنصری مانند $h_2 \in G$ وجود دارد که $h_2 \in H_2 \setminus H_1$.\\
اگر $h_1h_2 \in H_1$ باشد، آنگاه $h_2 \in H_1$ که فرض را نقض می‌کند. بنابراین، $H_1 \cup H_2 \neq G$.
\end{proof}

\end{solution}


\end{document}
