\documentclass[persian]{KNED}

\usepackage{amsmath}
\usepackage{xepersian}
\usepackage{xfrac}
\usepackage{amssymb}

\renewcommand{\KNEDRHdr}{عماد پورحسنی  - ۴۰۱۱۶۶۲۳}
\renewcommand{\KNEDLHdr}{روش های آماری / تکلیف چهارم}
\settextfont{Amiri}


\begin{document}

\begin{exercise}[آزمون فرضیه صفر برای میانگین جامعه]
آزمون فرضیه مربوط به میانگین جامعه را در حالتهای ۳و 4 همراه
با یک مثال بنویسید.\\
میدانیم که : 
\begin{enumerate}
    \item  حالت ۳ : هرگاه مشاهدات از توزیع نرمال با واریانس معلوم بدست آمده باشند، آنگاه
    \[
        \frac{\Bar{X} - \mu }{\sfrac{\sqrt{n}}{\sigma}} \sim N(0, 1)
    \]
    \item حالت ۴ : هرگاه مشاهدات از توزیع نرمال با واریانس مجهول بدست آمده باشند، آنگاه
    \[
        \frac{\Bar{X} - \mu }{\sfrac{\sqrt{n}}{S}} \sim t(n-1)
    \]

\end{enumerate}

\end{exercise}

\begin{solution}

\begin{enumerate}
    \item یک شرکت تولیدی لامپ‌های ال ای دی ادعا می‌کند که دارای روشنایی متوسط 800 لومن است. انحراف معیار آن 50 لومن است.یکی از مهندس های این شرکت مدعی شده است که لامپ های جدید این شرکت روشنایی متوسط 800 ندارند. یک تحقیق‌گر برای بررسی این ادعا، یک نمونه 62 عددی از لامپ‌های تولید شده را انتخاب کرده و روشنایی هر کدام را اندازه‌گیری کرده است. میانگین روشنایی این نمونه 728 لومن است. آیا ادعای شرکت درباره روشنایی متوسط لامپ‌های تولید شده توسط این شرکت را در سطح $\alpha = 0.06$ قبول می‌کنید؟\\
    \[ n = 62,
    \sigma = 50,
    \Bar{X} = 728,
    \]
    \[
\begin{cases}
    H_0: & \mu = 800 \\
    H_1: & \mu \neq 800
\end{cases}, \alpha = 0.06 \Rightarrow Z_{0.03} = 1.88.
\]
\[
R = \Big\{  \vert \frac{\Bar{X} - \mu_0}{\sfrac{\sqrt{n}}{\sigma}}\vert  > Z_{\frac{a}{2}} \Big\} \Rightarrow \vert \frac{\Bar{X} - \mu_0}{\sfrac{\sqrt{n}}{\sigma}} \vert = \vert \frac{728 - 800}{\sfrac{\sqrt{62}}{50}} \vert = 11.3386 > 1.88.
    \]
    شرایط ناحیه رد فرض صفر برقرار می باشد. این بدین معنا است که دلیلی بر پذیرش فرض صفر
وجود ندارد.\\
تفسیر : نظر این مهندس تایید شده و فرآیند تولید بازنگری میشود.\\
\rule{\linewidth}{1pt}
\item یک شرکت داروسازی ادعا می‌کند که داروی تسکین درد جدیدش میانگین زمان بهبودی از سرماخوردگی را به ۴ روز کاهش می‌دهد.سازمان نظارت و کنترل کیفیت از سوی دولت مدعی شده است که این دارو میانگین زمان بهبودی ۴ روز را ندارد. برای بررسی این ادعا، یک تحقیق‌گر یک نمونه تصادفی از ۵۰ نفر که از دارو استفاده کرده‌اند را انتخاب می‌کند و تعداد روزهایی که برای هر فرد بهبودی طول کشیده را ثبت می‌کند.با انحراف معیار نمونه $0.6$ روز. میانگین نمونه زمان بهبودی به $4.3$ روز یافت می‌شود، با سطح آزمون $\alpha = 0.05$، آیا می‌توانیم نتیجه بگیریم که این دارو واقعاً میانگین زمان بهبودی از سرماخوردگی را به ۴ روز کاهش می‌دهد؟
\[
n = 50, \quad
S = 0.6, \quad
\Bar{X} = 4.3,
\]
\[
\begin{cases}
    H_0: & \mu = 4 \\
    H_1: & \mu \neq 4
\end{cases}, \quad
\alpha = 0.05 \Rightarrow Z_{0.025} = 1.96.
\]
\[
R = \Big\{  \Big\vert \frac{\Bar{X} - \mu_0}{\sfrac{\sqrt{n}}{S}} \Big\vert  > Z_{\frac{\alpha}{2}} \Big\} \Rightarrow \Big\vert \frac{\Bar{X} - \mu_0}{\sfrac{\sqrt{n}}{S}} \Big\vert = \Big\vert \frac{4.3 - 4}{\sfrac{\sqrt{50}}{0.6}} \Big\vert =  2.3570 > 1.96.
\]

    شرایط ناحیه رد فرض صفر برقرار می باشد. این بدین معنا است که دلیلی بر پذیرش فرض صفر
وجود ندارد.\\
تفسیر : نظر این سازمان تایید میشود . شرکت مجبور است تبلیغات مبتنی بر کمتر از ۴ روز بودن زمان بهبودی را متوقف کند.\\

\end{enumerate}
\end{solution}

\end{document}
